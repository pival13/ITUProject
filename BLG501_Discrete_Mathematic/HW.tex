\documentclass{article}
\usepackage[T1]{fontenc}
\usepackage[english]{babel}
\usepackage{amsmath,amssymb}
\usepackage{changepage}
\usepackage{multicol}

\newcommand{\congru}[3]{#1 \equiv #2\ (\text{mod}\ #3)}

\begin{document}

\begin{center}
    \LARGE\textbf{BLG 501E – Discrete Mathematics}\\
    \vspace{1em}
    \Large\textbf{$\text{2}^{\text{nd}}$ Midterm Exam / Homework}\\
    \large\textit{Valentin L'HOMEL - 922110009}
\end{center}

\section{Residue classes}
\subsection{Calculate how many residue classes number 56 has.}

\begin{adjustwidth}{2em}{}
    $\varphi(56) = \varphi(7\times8)$\\
    \hspace*{2.7em}$= \varphi(7)\times\varphi(8)$\\
    \vspace{-0.5em}\\
    $\varphi(7) = 7-1 = 6$, because 7 is a prime number\\
    $\varphi(8) = \varphi(2^3) = 2^3(1-1/2) = 4$\\
    \vspace{-0.5em}\\
    $\varphi(56) = 6\times4 = 24$
\end{adjustwidth}

\vspace*{0.5em}\hspace*{-1.5em}56 has 24 residue classes.

\subsection{List the residue classes of 56 by using residue classes of 8 and 7. Write down your steps.}

The residue classes of 56 are all numbers which are equivalent to one of the member of the residue classes of 7 modulo 7 and equivalent to one of the member of the residue classes of 8 modulo 8.\\\vspace*{-0.7em}\\
\hspace*{1em}$S_{56} = \forall a\in S_7, \forall b\in S_8\ \{ x\ |\ \congru{x}{a}{7} \wedge \congru{x}{b}{8} \}$\\\vspace*{-0.7em}\\
Using the Chinese remainder theorem, we can write $x$ as the sum of one of the member of the residue classes of 7 times 8 times the inverse of 8 modulo 7 and one of the member of the residue classes of 8 times 7 times the inverse of 7 modulo 8.\\\vspace*{-0.7em}\\
\hspace*{1em}$\congru{x}{a\times8\times\overline{8}_7 + b\times7\times\overline{7}_8}{56}$\\\vspace*{0em}\\
\hspace*{2em}$S_7 = \{1,2,3,4,5,6\}$\\
\hspace*{2em}$S_8 = \{1,3,5,7\}$\\
\hspace*{2em}$\congru{8\times\overline{8}}{1}{7} \Rightarrow \overline{8}_7 = 1$\\
\hspace*{2em}$\congru{7\times\overline{7}}{1}{8} \Rightarrow \overline{7}_8 = 7$\\

\begin{tabular}{|c||c|c|c|c|c|c|}
    \hline
    $b\ \backslash\ a$ & 1 & 2 & 3 & 4 & 5 & 6 \\
    \hline\hline
    1 & 1 & 9 & 17 & 25 & 33 & 41 \\ 
    \hline
    3 & 43 & 51 & 3 & 11 & 19 & 27 \\
    \hline
    5 & 29 & 37 & 45 & 53 & 5 & 13 \\
    \hline
    7 & 15 & 23 & 31 & 39 & 47 & 55 \\
    \hline
\end{tabular}


\section{Prove Euler-Fermat theorem}

\begin{equation}
    \gcd(a,m) = 1\text{ and }\congru{a^{\varphi(m)}}{1}{m}
\end{equation}

Let's define two sets $S$ and $S'$ containing respectively the residue classes of $m$ and the residue classes of $m$ multiplied by $a$.

\vspace*{-0.5em}$$ S = \{x_1, x_2, ... x_{\varphi(m)}\} \hspace*{2em} S' = \{ax_1, ax_2, ... ax_{\varphi(m)}\} $$

We know that the elements of $S$ composed all possible numbers which are relatively primes to $m$. Because $a$ is also relatively prime to $m$, the elements of $S'$ are all coprimes with $m$ as well.

This implies that every element of $S'$ modulo $m$ are present in $S$.

$$ \forall i,j \in \{0,...\varphi(m)\}, \congru{ax_i}{x_j}{m} $$

$$ \congru{S'}{S}{m} $$\\\vspace*{-3.5em}\\
$$ \congru{\prod_i^{\varphi(m)}{ax_i}}{\prod_i^{\varphi(m)}{x_i}}{m} $$\\\vspace*{-3em}\\
$$ \congru{a^{\varphi(m)}\prod_i^{\varphi(m)}{x_i}}{\prod_i^{\varphi(m)}{x_i}}{m} $$

As every $x_i$ is relatively prime to $m$, we can safely removes them from the equations, resulting to $$ \congru{a^{\varphi(m)}}{1}{m} $$

\break\vfill
\section{Permutations}

\subsection{How many permutations of 6 people among 10 exists when a given person must be present}

This is equivalent to the number of permutations of 5 people among 9, times 6 (possible positions of the mandatory person).

$$ \frac{9!}{4!}\times6 = (9\times8\times7\times6\times5)\times6 = 90,720 $$

There are 90,720 different arrangements possible with 6 people among a group of 10, when one of them must be selected.

\subsection{How many permutations of 6 people among 10 exists when two given persons must be present}

This is equivalent to the number of permutations of 4 people among 8, times 30 (possible positions of both mandatory persons $6\times5$).

$$ \frac{8!}{4!}\times30 = (8\times7\times6\times5)\times30 = 50,400 $$

There are 50,400 different arrangements possible with 6 people among a group of 10, when two of them must be selected.

\subsection{How many permutations of 6 people among 10 exists when only one person among two on the group must be present, but not the other}

This is equivalent to the number of permutation of 5 people (randomly selected) among 8 (excluding the random prohibited persons), times 6 (possible position of the mandatory person) times 2 (possible mandatory person).

$$ \frac{8!}{3!}\times6\times2 = (8\times7\times6\times5\times4)\times12 = 80,640 $$

There are 80,640 different arrangements possible with 6 people among a group of 10, with two mutually exclusive persons, when one of those two must be included.

We can also consider it as the double of the substraction of the first two results (one mandatory person, but not two), which still produce the same result.


\break\vfill
\section{Prove the following equality}

\begin{equation}
    \sum_{k=0}^n{3^k\binom{n}{k}} = 4^n
\end{equation}

For $n=0$, we have the following:
$$ 3^0\binom{0}{0} = 1, \hspace*{3em} 4^0 = 1 $$

Now, let's assume the equation is true for a given value $n$, and prove it for $n+1$:

\begin{align}
    \sum_{k=0}^{n+1}{3^k\binom{n+1}{k}} &= \sum_{k=0}^{n+1}{3^k\left[\binom{n}{k}+\binom{n}{k-1}\right]}\\
    &= \sum_{k=0}^{n+1}{3^k\binom{n}{k}}+\sum_{k=0}^{n+1}{3^k\binom{n}{k-1}}\\
    &= \left[\sum_{k=0}^{n}{3^k\binom{n}{k}}+3^{n+1}\binom{n}{n+1}\right] + \left[3^0\binom{n}{-1}+\sum_{k=0}^{n}{3^{k+1}\binom{n}{k}}\right]\\
    &= (4^n+3^{n+1}\times0) + (1\times0+3\times4^n) = 4 \times 4^n\\
    \sum_{k=0}^{n+1}{3^k\binom{n+1}{k}} &= 4^{n+1}
\end{align}


\break\vfill
\section{How many ways can the digits 0,1,2,3,4,5,6,7,8,9 be arranged so that no even digit is in its original position}

We can find that by finding the total number of possibilities, and then removing every forbidden permutations.

The forbidden possibilies includes every cases were 0 is at position 0, 2 is at position 2 except when 0 is also at position 0 (as it has already been removed), and so on for 4, 6 and 8.\\
Using the inclusion-exclusion principle, we get:\vspace*{0.5em}

\hspace*{0.0em}$ |S_0| = |S_2| = ... \hspace*{4.4em}= 9! $\\\vspace*{-1em}\\
\hspace*{1.5em}$ |S_0 \cap S_2| = |S_0 \cap S_4| = ... = 8! $\\\vspace*{-1em}\\
\hspace*{1.5em}$ |S_0 \cap S_2 \cap S_4| = ...\hspace*{3em}= 7! $\\\vspace*{-1em}\\
\hspace*{1.5em}$ |S_0 \cap S_2 \cap S_4 \cap S_6| = ...\hspace*{0.9em}= 6! $\\\vspace*{-1em}\\
\hspace*{1.5em}$ |S_0 \cap S_2 \cap S_4 \cap S_6 \cap S_8|\hspace*{0.9em}= 5! $\\\vspace*{-1em}

$$\hspace{-13em}\left|\bigcup{S_{k}}\right|=\sum _{i=1}^{5}(-1)^{i-1}{\binom {5}{i}}(10-i)!$$
$$\hspace{-1.3em} = \binom{5}{1}9! - \binom{5}{2}8! + \binom{5}{3}7! - \binom{5}{4}6! + \binom{5}{5}5! $$
$$\hspace{4.6em} = 5\times362,880 - 10\times40,320 + 10\times5,040 - 5\times720 + 120 $$
$$\hspace{-19em}\left|\bigcup{S_{k}}\right|=1,458,120$$

The total number of ways to arrange the digits is therefore $10! - \left\lvert\bigcup{S_{k}}\right\rvert = 3,628,800 - 1,458,120 = 2,170,680$.

\end{document}