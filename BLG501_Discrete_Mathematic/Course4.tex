\documentclass{article}
\usepackage[T1]{fontenc}
\usepackage[english]{babel}
\usepackage{amsmath,amssymb}

\begin{document}

\newcommand{\congru}[3]{#1 \equiv #2\ (\text{mod}\ #3)}

\[
    \mathbf{test} \times 8
    \iff \to \gets \in \forall \exists \exists! \nexists
    \wedge \vee
    \mapsto
\]

\section{Ackermann function}

\[
    A(m, n) =
    \begin{cases}
        A(0, n) = n+1 \\
        A(m, 0) = A(m-1, 1) \\
        A(m, n) = A(m-1, A(m, n-1))
    \end{cases}
\]

\[
    A(1, n) =
    \begin{cases}
        A(0, A(1,n-1)) \\
        A(1, n-1) + 1 \\
        A(1, 0) + n \\
        A(0, 1) + n
    \end{cases}
    = n + 2
\]

\[
    A(1,n) = n+2 \\
    A(2,n) = 2n + 3 \\
    A(3,n) = 2^n - 3
\]

\section{Arithmetic}

\subsection{Division}

a divides b $\iff b = a . c$

$a | b \iff b/a \mod 0 == 0$

\subsection{Primes}

p is prime $\iff p > 1 \wedge \neg[\exists q(q \in {2,3...,p-1} \wedge q|p)]$

\subsubsection{Theorem of arithmetic}

$a \in N^+, a = p^{k_1}_1, ...$

\subsection{Division algorithm}

$a = d.q + r$

\subsection{Greastest Common Dividor}
\subsection{Least Common Multiple}

$lcm(a,b) = \frac{|a.b|}{gcd(a,b)}$

\section{Euclidean Algorithm (GCD)}

$gcd(a,b) = gcd(b,c) \iff a > b \wedge a \mod b = c$

$gcd(127, 37) = gcd(37, 16) = gcd(16, 5) = 1$

$gcd(127, 35)) = gcd(37, 16) = gcd(16, 5) = 1$

%Prove it => Prove that gcd(a,b) divides c
%                       gcd(b,c) divides a
%                    => gcd(a,b) = cd(b,c) <= gcd(b,c)
%                       gcd(b,c) = cd(a,b) <= gcd(a,b)
%                    => gcd(a,b) <= gcd(b,c) & gcd(b,c) <= gcd(ab) => gcd(a,b) == gcd(b,c)

\section{Congruence}

$\congru{a}{b}{m} \iff m | a-b \iff a = b + km \iff a\mod m = b\mod m$

$\overrightarrow{w} + \max  $

$\congru{ac}{bc}{m} \wedge gcd(c,m) = i \iff \congru{a}{b}{m/i}$

\section{Linear Diophantine Equations}

$ax + by = c \iff gcd(a,b) = d \wedge d|c$

$ax + by = c \wedge a,b,c,x,y \in \mathbf{I} \iff x = (sc + bk)/d, y = (tc - ak)/d \wedge d = gcd(a,b) \wedge as + bt = c$

$$ax + by = c \Rightarrow x = \frac{bk + cs}{\gcd(a,b)}, y = \frac{-ak + ct}{\gcd(a,b)} \iff \gcd(a,b) | c$$

\if false
172x + 20y = 1000
gcd(172, 20) = 4

4 = 12-8
4 = 2*12-20
4 = 2*(172-8*20)-20
4 = 2*172 - 17*20

s = 2, t = -17

x = (2000 + 20k) / 4 = 500 + 5k
y = (-17000 - 172k) / 4 = -4250 - 43k

x > 0 => k > -100
y > 0 => k < -98.84

k=-99

                    // 1, 0
                    // 0, 1
172 = 20 * 8 + 12,  // 1 - 8 * 0,   0 - 8 * 1
20 = 12 * 1 + 8     // 0 - 1 * 1,   1 - 1 * -8
12 = 8 * 1 + 4      // 1 - 1 * -1,  -8 - 1 * 9
8 = 4 * 2 + 0       // 2,           -17


\fi

\section{Linear modulo}

$\congru{ax}{b}{m}$

$gcd(a,m) = 1 \wedge \exists! \bar{a} \wedge 0 < \bar{a} < m \Rightarrow \congru{a\bar{a}}{1}{m}$

\if false
19x = 37 mod 141

141 = 7*19 + 8
19 = 2*8 + 3
8 = 2*3 + 2
3 = 1*2 + 1

1 = 3 - 2
1 = 3 - (8-2*3)
1 = -8 + 3*(19-2*8)
1 = 3*19 - 7*(141-7*19)
1 = 52*19 - 7*141

\bar{a} = 52

19*52*x = 37*52 mod 141
x = 91 mod 141
x = 91 + 141k
\fi

$gcd(a,m) = d \wedge$

\if false
ax = b (mod m) && gcd(a,b) = d
Prove congruence implies d|b

d | b
=> a = dk1, b = dk2, m = dk3, gcd(k1,k3) = 1
ax = b[m] => dk1x = dk2[dk3] => k1x = k2[k3]

ax = b[m] => ax - my = b
dk1x - dk2y = b
d(k1x-k2y) = b => d | b
\fi

$$ \congru{ax}{b}{m} \Rightarrow x = \frac{km + bs}{\gcd(a,m)} \iff \gcd(a,m) | b $$

\if false
28𝑥 ≡ 14 𝑚𝑜𝑑21
28 = 1*21 + 7
21 = 3*7
d = 7, s = 1, t = -1
x = 3k + 2
\fi

\section{Chinese remainder theorem}

$x = \sum_{i=1}^{n}{a_iM_iy_i} \wedge \congru{M_i y_i}{1}{m} \wedge m = \prod{m_i} \wedge M_i = m/m_i$

$$ \congru{a_nx}{b_n}{m_n} ... $$

\if false
15x = 21 mod 48
166x = 46 mod 22
x = 5 mod 13

5x = 7 mod 16
    16 = 5*3 + 1
        1 = 16 - 3*5
-3*5x = -3*7 mod 16
x = 11 mod 16

83x = 23 mod 11
    83 = 7*11 + 6
    11 = 1*6+5
    6 = 1*5+1
        1 = 2*83-15*11
2*83x = 2*23 mod 11
x = 2 mod 11
\fi

\end{document}